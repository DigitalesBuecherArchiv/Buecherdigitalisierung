
\section{Vorwort}

\subsection{Zielgruppe}

Diese Anleitung ist prinzipiell für all jene gedacht, welche sich
der ehrenvollen Aufgabe der Bücherdigitalisierung widmen und diese
auch möglichst effizient lösen wollen. Das Dokument soll sich vor
allem an informationstechnische Laien richten und strebt eine entsprechend
ausführliche Beschreibung an. Nichtsdestotrotz sollte sich jeder Interessent
der Tatsache bewusst sein, dass der Prozess der Bücherdigitalisierung
das Wissen über Informationstechnologien dringend vorraussetzt. Die
Selbstunterrichtung in gewisse Bereiche ist unabdingbar und in einer
digitalisierten Welt ohnehin von Nutzem.

Desweiteren bedarf es, um kein buntes Durcheinander entstehen zu lassen,
der Disziplin, den hier aufgestellten vereinheitlichten Prozess der
Digitalisierung einzuhalten. Werden diese hier gelehrten Arbeitsschritte
verinnerlichst, funktioniert das Digitalisieren wie von selbst.

\subsection{Aufbau}

Die Anleitung setzt sich aus Theorie und Praxis zusammen. Zuerst folgt
eine kurze Einführung in die Technologien und im Anschluss daran die
praktische Vorgehensweise. 
